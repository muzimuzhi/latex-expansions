% use color by color specification
\color[rgb]{0.8,0.85,1}

\DeclareRobustCommand\color{%
  \@ifnextchar[\@undeclaredcolor\@declaredcolor}


% expand \color
\@undeclaredcolor[rgb]{0.8,0.85,1}

\def\@undeclaredcolor[#1]#2{%
  \@ifundefined{color@#1}%
    {\c@lor@error{model `#1'}}%
    {\csname color@#1\endcsname\current@color{#2}%
     \set@color}%
  \ignorespaces}


% expand \@undeclaredcolor, #1 = "rgb", #2 = "0.8,0.85,1"
\csname color@rgb\endcsname\current@color{0.8,0.85,1}%
\set@color
\ignorespaces

% expand \csname
\color@rgb\current@color{0.8,0.85,1}%
\set@color
\ignorespaces

% from xetex.def, assume engine xetex is used
\def\color@rgb#1#2{\c@lor@@rgb#2\@@#1}
\def\c@lor@@rgb#1,#2,#3\@@#4{%
  \c@lor@arg{#1}%
  \c@lor@arg{#2}%
  \c@lor@arg{#3}%
  \edef#4{rgb #1 #2 #3}%
  }


% expand \color@rgb, #1 = "\current@color", #2 = "0.8,0.85,1"
\c@lor@@rgb0.8,0.85,1\@@\current@color
\set@color
\ignorespaces

% expand \c@lor@@rgb, #1 = "0.8", #2 = "0.85", 
%                     #3 = "1", #4 = "\current@color"
\c@lor@arg{0.8}%
\c@lor@arg{0.85}%
\c@lor@arg{1}%
\edef\current@color{rgb 0.8 0.85 1}%
\set@color
\ignorespaces

% from xetex.def
\def\c@lor@arg#1{%
  \dimen@#1\p@
  \ifdim\dimen@<\z@\dimen@\maxdimen\fi
  \ifdim\dimen@>\p@
    \PackageError{color}{Argument `#1' not in range [0,1]}\@ehd
  \fi}
\def\set@color{%
    \special{color push  \current@color}%
    \aftergroup\reset@color}
\def\reset@color{\special{color pop}}

% skip three \c@lor@arg
% expand \edef
% expand \set@color, \reset@color
\current@color = "rgb 0.8 0.85 1"
\special{color push rgb 0.8 0.85 1}%
\aftergroup{\special{color pop}}
\ignorespaces


% =========================================
% define new named color
\definecolor{light-blue}{rgb}{0.8,0.85,1}

\def\definecolor#1#2#3{%
  \@ifundefined{color@#2}%
    {\c@lor@error{model `#2'}}%
    {\@ifundefined{\string\color @#1}{}%
      {\PackageInfo{color}{Redefining color #1}}%
     \csname color@#2\expandafter\endcsname
         \csname\string\color @#1\endcsname{#3}}}

% expand \definecolor, #1 = "light-blue", #2 = "rgb", #3 = "0.8,0.85,1"
% skip two \@ifundefined
\@ifundefined{\string\color @#1}{}%
  {\PackageInfo{color}{Redefining color #1}}%
\csname color@#2\expandafter\endcsname
   \csname\string\color @#1\endcsname{#3}

% expand \csname
\color@rgb"\\color@light-blue"{0.8,0.85,1}

\edef"\\color@light-blue"{rgb 0.8 0.85 1}%

% =========================================
% use named color
\color{light-blue}

% expand \color
\@declaredcolor{light-blue}

\def\@declaredcolor#1{%
  \@ifundefined{\string\color @#1}%
    {\c@lor@error{`#1'}}%
    {\expandafter\let\expandafter\current@color
     \csname\string\color @#1\endcsname
     \set@color}%
  \ignorespaces}


% expand \@declaredcolor, #1 = "light-blue"
% \@ifundefined is false
\expandafter\let\expandafter\current@color
 \csname\string\color @light-blue\endcsname
 \set@color
\ignorespaces

% expand \expandafter
\let\current@color"\\color@light-blue"
\set@color
\ignorespaces


% =========================================
% use pdftex engine
\color[rgb]{0.8,0.85,1}

% from pdftex.def
\def\color@rgb#1#2{\c@lor@@rgb#2\@@#1}% the same as in "xetex.def"
\def\c@lor@@rgb#1,#2,#3\@@#4{%
  \c@lor@arg{#1}%
  \c@lor@arg{#2}%
  \c@lor@arg{#3}%
  \edef#4{#1 #2 #3 rg #1 #2 #3 RG}%
  }
\def\set@color{%
  \pdfcolorstack\@pdfcolorstack push{\current@color}%
  \aftergroup\reset@color}
\def\reset@color{\pdfcolorstack\@pdfcolorstack pop\relax}
\chardef\main@pdfcolorstack=0 %
\@ifundefined{@pdfcolorstack}
  {\def\@pdfcolorstack{\main@pdfcolorstack}}{}


% expand till \c@lor@@rgb
\c@lor@@rgb0.8,0.85,1\@@\current@color
\set@color
\ignorespaces

% expand \c@lor@@rgb, #1 = "0.8", #2 = "0.85", 
%                     #3 = "1", #4 = "\current@color"
\c@lor@arg{0.8}%
\c@lor@arg{0.85}%
\c@lor@arg{1}%
\edef\current@color{0.8 0.85 1 rg 0.8 0.85 1 RG}%
\set@color
\ignorespaces

% skip three \c@lor@arg
% expand \set@color
\current@color = "0.8 0.85 1 rg 0.8 0.85 1 RG"
\pdfcolorstack 0 push{0.8 0.85 1 rg 0.8 0.85 1 RG}%
\aftergroup{\pdfcolorstack0 pop\relax}


% =============================================
% summary

% xetex
\current@color  = "rgb 0.8 0.85 1"
\set@color      = \special{color push \current@color}
\reset@color    = \special{color pop}

% pdftex
\current@color  = "0.8 0.85 1 rg 0.8 0.85 1 RG"
\set@color      = \pdfcolorstack 0 push{\current@color}
\reset@color    = \pdfcolorstack 0 pop\relax
