\documentclass{article}
%\usepackage{trimspaces}

\usepackage{unravel}
\unravelsetup{max-action=1000, max-input=1000, max-output=200}
\long\def\beginunravel#1\endunravel{\unravel{#1}}
\def\endunravel{}

\makeatletter
\catcode`\Q=3
\newcommand\trim@spaces[1]{%
  \romannumeral-`\q\trim@trim@\noexpand#1Q Q%
}
\long\def\trim@trim@#1 Q{\trim@trim@@#1Q}
\long\def\trim@trim@@#1Q#2{#1}
\catcode`\Q=11
\makeatother


\begin{document}
\ttfamily

\makeatletter
\leavevmode
\beginunravel
%\tracingall
\trim@spaces{ a }
%\tracingnone
\endunravel
\makeatother
\end{document}

%% start
\trim@spaces{␣a␣}

\newcommand\trim@spaces[1]{%
  \romannumeral-`\q\trim@trim@\noexpand#1Q Q%
}
% expand \trim@spaces, #1 = "␣a␣"
\romannumeral-`\q\trim@trim@\noexpand␣a␣Q Q%

% expand \romannumeral, got a minus sign and an integer denotation `\q
%   keep on expanding to find the trailing optional space
% See "texdoc texbytopic", sec. 7.2.2 Denotations: characters and
%   sec. 7.2.7 Trailing spaces.
\romannumeral-`\q\trim@trim@@\noexpand␣a␣Q Q

\long\def\trim@trim@#1 Q{\trim@trim@@#1Q}
% expand \trim@trim@, #1 = "\noexpand␣a" (delimiter "␣Q" dropped)
\romannumeral-`\q\trim@trim@@\noexpand␣aQ Q

\long\def\trim@trim@@#1Q#2{#1}
% expand \trim@trim@@, #1 = "\noexpand␣a" (delimiter "Q" dropped), #2 = "Q"
%   #2 is undelimited, thus the space right before it is skipped
\romannumeral-`\q\noexpand␣a

% expand \noexpand
\romannumeral-`\q␣a

% "\romannumearl-`\q␣" expands to nothing
%   "-`\q" denotes a negative integer "-113" (113 is the character code of q)
%   "\romannumeral<negative integer>" expands to nothing
a
%% done

%% start
\trim@spaces{a}

\newcommand\trim@spaces[1]{%
  \romannumeral-`\q\trim@trim@\noexpand#1Q Q%
}
% expand \trim@spaces, #1 = "a"
\romannumeral-`\q\trim@trim@\noexpand aQ Q%

% expand \romannumeral
\romannumeral-`\q\trim@trim@@\noexpand aQ Q

\long\def\trim@trim@#1 Q{\trim@trim@@#1Q}
% expand \trim@trim@, #1 = "\noexpand aQ" (delimiter " Q" dropped)
\romannumeral-`\q\trim@trim@@\noexpand aQ

% expand \trim@trim@@, #1 = "\noexpand a", #2 = "Q"
\romannumeral-`\q\noexpand a

% expand \noexpand
%
% If "a" is expandable, "\noexpand a" expands to a temp, internal unexpandable
% token "\notexpanded:a".
% See "texdoc interface3", sec. 24.7 Description of all possible tokens.
\romannumeral-`\q a

% "\romannumearl-`\q" expands to nothing
a
%% done
